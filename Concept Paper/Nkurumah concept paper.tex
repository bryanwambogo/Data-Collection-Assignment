\documentclass[12pt, a4paper]{article}
\pagenumbering{arabic}
\usepackage[utf8]{inputenc}
\usepackage[T1]{fontenc}
\usepackage{geometry}
\usepackage{graphicx}
\graphicspath{{images/}}
\geometry{a4paper}
\usepackage{helvet}
\newcommand\tab[1][1cm]{\hspace*{#1}}
\renewcommand{\familydefault}{\sfdefault}
\setlength{\topmargin}{-2cm}
\setlength{\oddsidemargin}{0cm}
\setlength{\textheight}{24cm}
\setlength{\textwidth}{16cm}
\usepackage{graphicx}
\usepackage{listings}
\usepackage{sectsty}
%Use Helvetica as the sansserif font
\usepackage{helvet}
%Use sffamily for all titles
\allsectionsfont{\sffamily}
\title{THE EXPERIENCE OF SERVICES BY STUDENTS IN NKURUMAH HALL IN MAKERERE UNIVERSITY}
\author{Prepared by Wambogo Brian}
\date{$19^{th} May$ $2017$}
\begin{document}
\maketitle
\clearpage
\section{Introduction}
\subsection{Research background}
Nkurumah hall is a place of a residence within Makerere University that was built in 1954, it was named after the late fist president of Ghana and the founder of Pan African movement called Nkwame Nkurumah. It was first under the management of Northcote hall currently called Nsibirwa hall and it was also referred to as a colony by the residents of Nsibirwa hall. Nkurumah hall only admits and assigns rooms to male students who commonly refer to themselves as activists. Ever since the hall became independent,there has always been a lot of criticism and unsatiety by the students on the services provided by the hall ranging from the amenities like (electricity and water), food and security due to embezzlement of funds and poor budgeting by the management of the hall.
\subsection{Problem statement}
In the past years up to the current year, there has always been a lot of complaints on the way services are delivered to students. This comes as a way of shortage of water in the hall, sub-standard meals, poor sanitation and insecurity that often results into theft of students’ property. This can be solved by having a proper budget of funds to allocate for the above services and having a corrupt–free and responsible management in the hall premises.
\subsection{Aim and objectives}
\subsubsection{Aim or general objective}
To assess or analyse the quality of services provided to students.
\subsubsection{Specific objectives}
To compare  the accommodation fee paid by the students and the rooms they are offered to sleep.
To determine the type of meals offered to students, also in comparison to the food fee they pay to the hall management.
To determine the general hygiene in the residence, the rate of water supply and the availability of security in the hall.


\subsection{Research significance}
The purpose of this study is to find out the impact of the services provided by the hall on the students and to find out the solutions that can be undertaken in order to improve the services for a comfortable better living by the students.




\subsection{Research scope}
This study focuses on the type and quality of services  provided by the hall and the contentment of the students by the services.
\section{Methodology}
This research is going to be carried out through making interviews with students, use of questionnaires, recording down important information on ground and the use of  the  ODK collect application to capture videos, images and audios about the current situation in the hall.
\clearpage

\Large{\textbf{References}}\\
\normalsize Campusbee.com\\
Students chairman Nkurumah hall (Ahmed Salim)


\end{document}